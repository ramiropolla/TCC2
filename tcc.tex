\documentclass{article}
\usepackage[utf8]{inputenc}
\usepackage{amsmath}

\author{Eu mesmo}
\date{Hoje}
\title{Uma nova abordagem para a troca de
       lampadas elétricas: utilizando um banquinho}

\newcommand{\vetor}[1]{\textbf{#1}}

\begin{document}

\maketitle

\newpage

\section{Introdução}
\label{sec:intro}

Neste trabalho  iremos apresentar uma nova abordagem
para a troca  de lampadas elétricas.
Subindo  num  banquinho. % hehehe

% aqui eu tenho que melhorar a introducao

\section{Motivação}
\label{sec:motiv}

Pessoas muito baixas não conseguem trocar lampadas.
O presente trabalho sugere que elas subam num
banquinho para alcançar a lampada.

Como vimos na introducao (secao \ref{sec:intro}),
...

Uma fórmula qualquer: $V=R I$

$$I = \frac{V}{R}$$

$$0=1+e^{j\pi}$$

IDFT (fórmula numero \ref{eq:idft}):

\begin{equation}
\label{eq:idft}
x_n = \frac{1}{N}
\sum_{k=0}^{N-1} X_k \cdot
e^{i2\pi kn/N}
\end{equation}

\begin{align}
\nabla \cdot \vetor{E} & = \frac{\rho}{\epsilon_0} \\
\nabla \cdot \vetor{B} & = 0 \\
\nabla \times \vetor{E}
& = - \frac{\partial \vetor{B}}{\partial t}
\end{align}

\end{document}